\section{Probing Gravity With SN~Ia  Peculiar Velocity Surveys}
In the late 1990's, Type~Ia supernovae (SNe~Ia) were used as distance probes to measure the homogeneous expansion history of the Universe.  The remarkable discovery
that the expansion is accelerating  has called into question our basic understanding of the gravitational forces within the Universe.  Either it
is dominated by a ``dark energy'' that is gravitationally repulsive, or General Relativity is inadequate and needs to be replaced by a modified theory of
gravity.  It is only appropriate that in the upcoming decade, with their sheer numbers, solid-angle coverage,
and improved distance precisions, SNe~Ia will provide measurements of the {\it inhomogeneous} motions of structures in the Universe
that will provide an unmatched test of whether dark energy or modified gravity is responsible for the accelerating expansion of the Universe.

In the next decade, SNe~Ia will be used as peculiar-velocity probes to measure  the influence of gravity on structure formation within the Universe.
Peculiar velocities induce scatter along the redshift axis of the SN Hubble diagram, which is
pronounced at low redshifts and when the magnitude scatter (e.g.\ due to intrinsic magnitude dispersion) is small.
The peculiar velocity power spectrum is sensitive to the growth of structure as $P_{vv}\propto (fD)^2$, where $D$ is  the spatially-independent
``growth factor'' in the linear evolution of density perturbations and
$f \equiv \frac{d\ln{D}}{d\ln{a}}$ is the linear growth rate where $a$ is the scale factor  \citep{2006PhRvD..73l3526H,2011ApJ...741...67D}.
The combination $fD$ evolves with redshift;
%\footnote{
%To be precise, the peculiar
%velocity power spectrum also depends on the Hubble parameter as $P_{vv}\propto (HfD)^2$.  A supernova  survey  measures luminosity distance fluctuations
%$\delta_{d_L} = (d_L- \bar{d}_L(z))/\bar{d}_L(z)$, where $d_L$ and  $\bar{d}_L(z)$ are the measured and expected distances at the observed redshift $z$.
%To first order in peculiar velocity along the line of sight $v$, $\delta_{d_L} =  v \left(1-\frac{1}{H\bar{d}_L(z)}\right) \approx -\frac{v}{H\bar{d}_L(z)}$
%at low redshift.   The $H$-dependences of  $P_{vv}$ and the conversion from distances to velocities cancel, making  peculiar velocity surveys sensitive to $(fD)^2$.
%}
the $\Lambda$CDM prediction for the $z=0$ peculiar velocity power spectrum is shown in Figure~\ref{power:fig}.

The  growth of structure depends on gravity;
\citet{PhysRevD.72.043529,2007APh....28..481L} find that General Relativity, $f(R)$,  and DGP gravity follow the relation
$f \approx \Omega_M^\gamma$ with the growth index $\gamma=0.55, 0.42, 0.68$ respectively \citep[see][for a review
or these  models]{HUTERER201523}.  
Using this parameterization, peculiar velocity
surveys probe  gravity through by modeling $fD=\Omega_M^{\gamma} \exp{\left(-\int_a^1 \Omega_M^{\gamma} d\ln{a} \right)}$,
where $\Omega_M(a)$ also depends on the gravity model.
The  $\gamma$-dependence of $fD(z)$ is shown 
in Figure~2 of  \citet{1475-7516-2013-04-031}.

%
%The parameter $\sigma_8$, the  standard deviation of overdensities in 8$h^{-1}$Mpc spheres, is 
%commonly used in place of $D$ to normalize the
%overall amplitude of  overdensities, so the standard parameterization used by the community is $f\sigma_8$.}


\begin{figure}[h]
\centering
\includegraphics[width=0.6\textwidth]{/Users/akim/project/PeculiarVelocity/doc/src/new2.png}
%\includegraphics[width=0.5\textwidth]{../outcosmo/zmax_.png}
\caption{Dimensionless peculiar velocity power spectrum $k^3P_{vv}(z=0)$ for $\mu \equiv \cos{(\hat{k} \cdot \hat{r})}=1, 0.5$ (magenta, cyan) 
where $\hat{r}$ is the line of sight, as predicted for General Relativity in the linear regime.
Overplotted are peculiar-velocity power-spectrum shot noise  (diagonal lines) for various observing parameters.  Red shows the shot noise expected from a 2-year LSST survey
while black shows a 10-year LSST survey.  The dotted and dashed lines indicate the assumed intrinsic magnitude dispersion, using 0.08 (dashed) or 0.15 mag (dotted).  The expected shot
noise from TAIPAN is shown in green (dash-dotted). 
%
%Overplotted are volume-weighted peculiar velocity shot-noise  $k^3\sigma^2/n$ at $z=0.1$ expected from 2- and 10-year (red, black) supernova densities, 0.08 and 0.15~mag (dashed, dotted) intrinsic magnitude dispersions, and TAIPAN (green, dash-dotted).
The bottom solid grey horizontal lines show the approximate range of $k$ expected to be used in surveys with corresponding
redshift depths $z_{\rm max}$.
\label{power:fig}}
\end{figure}

The same SNe~Ia used  to measure peculiar velocities can also serve as tracers of mass overdensities.  Overdensities and their motions are connected by the
continuity equation, so the SN-overdensity power spectrum also depends on gravity 
as $P_{\delta \delta }\propto (bD + fD\mu^2)^2$ where $b$ is the SN bias and $\mu\equiv \cos{(\hat{k} \cdot \hat{r})}$ where $\hat{r}$ is the direction of
the line of sight.  
The bias is a ``nuisance'' parameter, not present in the velocity power spectrum, which must marginalized out when inferring $\gamma$.
The same field is responsible for both overdensities and velocities so when combined in a common analysis the sample variance limit is lowered.


A SN~Ia peculiar velocity survey is composed of several components:
\begin{itemize}
\item Transient Discoveries -- Wide-field cadenced imaging surveys detect and localize the angular coordinates of new supernovae.
\item Discovery Screening -- Galaxy redshift catalogs, supplemental imaging data can provide subsets of the discoveries for classification
using relatively expensive follow-up resources.
\item SN~Ia Classification -- Spectroscopy classifies the SN~Ia from among the pool of  transient discoveries. 
\item (Host-galaxy) Redshift -- Spectroscopy provides redshifts. 
Host-galaxies generally have more/sharper features that provide more precise redshifts than the supernovae themselves. Resolution $R>200$
is sufficient to ensure that statistical uncertainties are negligible.
\item SN~Ia Distance -- Imaging produces photometry and colors for light-curve fitting and getting distances.  The transient search naturally provides some data
for this, of quality that varies depending on the survey strategy.
Spectroscopy and NIR photometry
have been shown to provide more precise distances than with optical light curves alone.
\end{itemize}
The above suite of observations provide a pure sample of SNe~Ia with observed redshifts and cosmological redshifts (inferred from the distances and
the background Hubble law)
whose difference is the radial peculiar velocity.

The precision  in measuring $\gamma$ can be projected for different peculiar velocity surveys.
The primary  parameters that affect the precision are: solid angle $\Omega$, SN number density $n$, source intrinsic
magnitude dispersion $\sigma_M$, and for a distance-limited survey the maximum distance $r_{\text{max}}$ (alternatively redshift $z_{\text{max}}$).
The dependence is most simply discerned 
in the  Fisher information matrix
\begin{align}
F_{ij} 
%& = \frac{V}{2}\int \frac{d^3k}{(2\pi)^3} \text{Tr}\left[ C^{-1} \frac{\partial C}{\partial \lambda_i} C^{-1}
%\frac{\partial C}{\partial \lambda_j} \right]\\
& = \frac{\Omega}{8\pi^2} \int_{r_{\rm min}}^{r_{\rm max}}  \int_{k_{\rm min}}^{k_{\rm max}}  \int_{-1}^{1} r^2 k^2 \text{Tr}\left[ C^{-1} \frac{\partial C}{\partial \lambda_i} C^{-1}
\frac{\partial C}{\partial \lambda_j} \right] d\mu\,dk\,dr
\label{fisher:eqn}
\end{align}
where
\begin{equation}
C(k,\mu,a)  =
  \begin{bmatrix}
   P_{\delta \delta}(k,\mu,a) + \frac{1}{n} &
   P_{v\delta}(k,\mu,a)  \\
   P_{v\delta}(k,\mu,a)  &
  P_{vv}(k,\mu,a) + \frac{\sigma^2}{n}
   \end{bmatrix},
\label{cov:eq}
\end{equation}
and the
peculiar-velocity  uncertainty  ($\sigma$) is related to magnitude uncertainty via $\sigma_M^2 =  \left(\frac{5}{\ln{10}}\right)^2 \left(1-\frac{1}{Ha\chi}\right)^2 \sigma^2$.
The dependences of $\gamma$ and other parameters enter through $fD$ in the relations $P_{vv}\propto (fD\mu)^2$, the SN~Ia host-galaxy count overdensity
power spectrum $P_{\delta \delta }\propto (bD + fD\mu^2)^2$, and the galaxy-velocity cross-correlation $P_{vg}
\propto  (bD + fD\mu^2)fD\mu$.  
 The density and velocity covariances depend on the parameter set $\lambda \in \{\gamma, \Omega_{M0}, b\}$.
Taking $\Lambda$CDM as our fiducial model, 
$\Omega_M=\frac{\Omega_{M_0}}{\Omega_{M_0} + (1-\Omega_{M_0})a^3}$.  
The uncertainty in the growth index is bounded by $\sqrt{F^{-1}_{\gamma \gamma}}$.