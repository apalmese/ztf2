\documentclass[11pt, oneside]{article}   	% use "amsart" instead of "article" for AMSLaTeX format
\usepackage{geometry}                		% See geometry.pdf to learn the layout options. There are lots.
\geometry{letterpaper}                   		% ... or a4paper or a5paper or ... 
%\geometry{landscape}                		% Activate for rotated page geometry
%\usepackage[parfill]{parskip}    		% Activate to begin paragraphs with an empty line rather than an indent
\usepackage{graphicx}				% Use pdf, png, jpg, or eps§ with pdflatex; use eps in DVI mode
								% TeX will automatically convert eps --> pdf in pdflatex		
\usepackage{amssymb,amsmath}
\usepackage{natbib,verbatim}
\usepackage{hyperref}
\usepackage{aas_macros}


\title{LBL Peculiar-Velocity Program: ZTF2 and LSST}
\author{The Author}
%\date{}							% Activate to display a given date or no date

\begin{document}
\maketitle


%\section{}
%\subsection{}
Measuring peculiar velocities using Type~Ia supernovae is a primary research topic in the upcoming decade.

A new group is forming ZTF2, a new 3-year survey using the existing ZTF infrastructure.  We present the case for LBL joining ZTF2.

\section{Probing Gravity With SN~Ia  Peculiar Velocity Surveys}
In the late 1990's, Type~Ia supernovae (SNe~Ia) were used as distance probes to measure the homogeneous expansion history of the Universe.  The remarkable discovery
that the expansion is accelerating  has called into question our basic understanding of the gravitational forces within the Universe.  Either it
is dominated by a ``dark energy'' that is gravitationally repulsive, or General Relativity is inadequate and needs to be replaced by a modified theory of
gravity.  It is only appropriate that in the upcoming decade, with their sheer numbers, solid-angle coverage,
and improved distance precisions, SNe~Ia will provide measurements of the {\it inhomogeneous} motions of structures in the Universe
that will provide an unmatched test of whether dark energy or modified gravity is responsible for the accelerating expansion of the Universe.

In the next decade, SNe~Ia will be used as peculiar-velocity probes to measure  the influence of gravity on structure formation within the Universe.
Peculiar velocities induce scatter along the redshift axis of the SN Hubble diagram, which is
pronounced at low redshifts and when the magnitude scatter (e.g.\ due to intrinsic magnitude dispersion) is small.
The peculiar velocity power spectrum is sensitive to the growth of structure as $P_{vv}\propto (fD)^2$, where $D$ is  the spatially-independent
``growth factor'' in the linear evolution of density perturbations and
$f \equiv \frac{d\ln{D}}{d\ln{a}}$ is the linear growth rate where $a$ is the scale factor  \cite{2006PhRvD..73l3526H,2011ApJ...741...67D}.
%\footnote{
%To be precise, the peculiar
%velocity power spectrum also depends on the Hubble parameter as $P_{vv}\propto (HfD)^2$.  A supernova  survey  measures luminosity distance fluctuations
%$\delta_{d_L} = (d_L- \bar{d}_L(z))/\bar{d}_L(z)$, where $d_L$ and  $\bar{d}_L(z)$ are the measured and expected distances at the observed redshift $z$.
%To first order in peculiar velocity along the line of sight $v$, $\delta_{d_L} =  v \left(1-\frac{1}{H\bar{d}_L(z)}\right) \approx -\frac{v}{H\bar{d}_L(z)}$
%at low redshift.   The $H$-dependences of  $P_{vv}$ and the conversion from distances to velocities cancel, making  peculiar velocity surveys sensitive to $(fD)^2$.
%}
The $\Lambda$CDM prediction for the $z=0$ peculiar velocity power spectrum is shown in Figure~\ref{power:fig}. The growth of structure depends on gravity;
\cite{2007APh....28..481L} find that General Relativity, $f(R)$,  and DGP gravity follow the relation
$f \approx \Omega_M^\gamma$ with $\gamma=0.55, 0.42, 0.68$ respectively (see \cite{HUTERER201523} for a review
or these  models).  Using this parameterization to model gravity, peculiar velocity
surveys probe $\gamma$ through $fD$, whose $\gamma$-dependence is plotted 
in Figure~2 of  \cite{1475-7516-2013-04-031}.
%$fD=\Omega_M^{\gamma} \exp{\left(\int_a^1 \Omega_M^{\gamma} d\ln{a} \right)}$
%
%The parameter $\sigma_8$, the  standard deviation of overdensities in 8$h^{-1}$Mpc spheres, is 
%commonly used in place of $D$ to normalize the
%overall amplitude of  overdensities, so the standard parameterization used by the community is $f\sigma_8$.}


\begin{figure}[h]
\centering
\includegraphics[width=0.49\textwidth]{/Users/akim/project/PeculiarVelocity/doc/src/new2.png}
%\includegraphics[width=0.5\textwidth]{../outcosmo/zmax_.png}
\caption{Volume-weighted peculiar velocity power spectrum $k^3P_{vv}(z=0)$ for $\mu \equiv \cos{(\hat{k} \cdot \hat{r})}=1, 0.5$ (magenta, cyan) 
where $\hat{r}$ is the line of sight, as predicted for General Relativity in the linear regime.
Overplotted are peculiar-velocity power-spectrum shot noise  (diagonal lines) for various observing parameters.  Red shows the shot noise expected from a 2-year LSST survey
while black shows a 10-year LSST survey.  The dotted and dashed lines indicate the assumed intrinsic magnitude dispersion, using 0.08 (dashed) or 0.15 mag (dotted).  The expected shot
noise from TAIPAN is shown in green (dash-dotted). 
%
%Overplotted are volume-weighted peculiar velocity shot-noise  $k^3\sigma^2/n$ at $z=0.1$ expected from 2- and 10-year (red, black) supernova densities, 0.08 and 0.15~mag (dashed, dotted) intrinsic magnitude dispersions, and TAIPAN (green, dash-dotted).
The bottom solid grey horizontal lines show the approximate range of $k$ expected to be used in surveys with corresponding
redshift depths $z_{\rm max}$.
\label{power:fig}}
\end{figure}

The same SNe~Ia used  to measure peculiar velocities can also serve as tracers of mass overdensities.  Overdensities and their motions are connected by the
continuity equation, so the SN-overdensity power spectrum also depends on gravity 
as $P_{\delta \delta }\propto (bD + fD\mu^2)^2$ where $b$ is the SN bias and $\mu\equiv \cos{(\hat{k} \cdot \hat{r})}$ where $\hat{r}$ is the direction of
the line of sight.  
The bias is a ``nuisance'' parameter, not present in the velocity power spectrum, which must marginalized out when inferring $\gamma$.
The same field is responsible for both overdensities and velocities so when combined in a common analysis the sample variance limit is lowered.


\section{Near- and Long-Term Projections}

Peculiar velocity surveys have already been  used to measure
the effective $fD$ in redshift bins  (referred to as $f\sigma_8$), though not to a level where  gravity models can be precisely distinguished.
 \cite{2017MNRAS.471..839A} use 6dFGS peculiar velocities using  Fundamental Plane distances of elliptical galaxies to estimate absolute magnitudes
 with
 $\sim 0.43$~mag  precision, yielding a 15\% uncertainty in $fD$ at $z\approx 0$.
The upcoming 
TAIPAN survey \cite{2017PASA...34...47D} will obtain Fundamental Plane galaxies with densities of $n_g \sim 10^{-3}h^3$\,Mpc$^{-3}$,
and the WALLABY+WNSHS surveys \cite{2008ExA....22..151J} will obtain Tully-Fisher distances (based on the $\sim 0.48$~mag calibration of absolute magnitude based on the  HI 21cm line width)
of galaxies with densities $n_g \sim 2\times 10^{-2} - 10^{-4} h^3$\,Mpc$^{-3}$ from
$z=0-0.1$ covering 75\% of the sky.
These surveys combined are projected to have 3\% uncertainties in $fD$ \cite{2017MNRAS.464.2517H}.
For reference, DESI projects a 10\% precision of $fD$ at $z \approx 0.3$  by looking 
for signatures (Redshift Space Distortions; RSD) expected from galaxies infalling toward mass overdensities.
Relative to galaxies with  Fundamental Plane or Tully-Fisher distances, 
SN~Ia host galaxies currently have significantly lower number density but have better per-object peculiar velocity precision.
Existing SN~Ia samples
have been used to test and ultimately find spatial correlations in peculiar velocities that may be attributed to the growth of structure
\cite{PhysRevLett.99.081301,2008MNRAS.389L..47A,2014MNRAS.444.3926J,2015JCAP...12..033H, 2017JCAP...05..015H}.
SNe~Ia discovered by ASAS-SN, ATLAS, and ZTF \cite{2014ApJ...788...48S,2018PASP..130f4505T,2019PASP..131a8002B} over the next several years will provide first probative measures of $fD$ at $z<0.1$.
%Measurement of the velocity field using LSST-discovered
%SNe~Ia has been quantified by \cite{2011PhRvD..83d3004B,2017JCAP...01..060O}


Two advances in the upcoming decade will make SN~Ia peculiar velocities more powerful.
First, the precision of SN~Ia distances can be improved.  The commonly-used empirical 2-parameter spectral model yields  absolute magnitude
dispersion $\sigma_M \gtrsim 0.12$~mag.  However, SNe transmit more information than just the light-curve shape and single color used in current SN models.
Recent studies indicate that with the right data, SN absolute
magnitudes can be calibrated to $\sigma_M \lesssim 0.08$ mag \cite[see e.g.][]{2012MNRAS.425.1007B, 2015ApJ...815...58F}. 
Though not yet
established, it is anticipated that such a reduction in intrinsic dispersion comes with a reduction in the magnitude bias correlated with host-galaxy properties
that is observed using current calibrations.  At this precision the intrinsic velocity dispersion  at $z=0.028$ is  $300$~km\,s$^{-1}$, i.e.\ a single SN~Ia  is of such quality as to
measure a peculiar velocity with $S/N \sim 1$.
 If corrections of all SNe~Ia are not possible, the use of SN~Ia subclasses is an option though at the expense of reducing the
numbers of velocity probes.
Secondly,  in the upcoming decade cadenced wide-field imaging surveys such as ZTF2 and LSST
  will increase the number of identified  $z<0.2$ Type~Ia supernovae from the hundreds to the
hundreds of thousands; over the course of 10-years, LSST will find $\sim150,000$ $z<0.2$ SNe~Ia
 for which good light curves can be measured, corresponding to a  number density of $n \sim 5\times 10^{-4}h^3$\,Mpc$^{-3}$.
  This sample has comparable
 number density and more galaxies at deeper redshifts than projected by WALLABY and TAIPAN.  With similar densities,
 the (two) ten-year SN~Ia survey will have
 a (6) 29$\times$ reduction in shot-noise, $\sigma^2_M/n$, relative to the Fundamental Plane survey of TAIPAN.



The primary  parameters that affect the precision obtained from a peculiar velocity survey are: solid angle $\Omega$, number density $n$, source intrinsic
magnitude dispersion $\sigma_M$, and for a distance-limited survey the maximum distance $r_{\text{max}}$ (alternatively redshift $z_{\text{max}}$).
The dependence is most simply discerned 
in the  Fisher information matrix
\begin{align}
F_{ij} 
%& = \frac{V}{2}\int \frac{d^3k}{(2\pi)^3} \text{Tr}\left[ C^{-1} \frac{\partial C}{\partial \lambda_i} C^{-1}
%\frac{\partial C}{\partial \lambda_j} \right]\\
& = \frac{\Omega}{8\pi^2} \int_{r_{\rm min}}^{r_{\rm max}}  \int_{k_{\rm min}}^{k_{\rm max}}  \int_{-1}^{1} r^2 k^2 \text{Tr}\left[ C^{-1} \frac{\partial C}{\partial \lambda_i} C^{-1}
\frac{\partial C}{\partial \lambda_j} \right] d\mu\,dk\,dr
\label{fisher:eqn}
\end{align}
where
\begin{equation}
C(k,\mu)  =
  \begin{bmatrix}
   P_{\delta \delta}(k,\mu) + \frac{1}{n} &
   P_{v\delta}(k,\mu)  \\
   P_{v\delta}(k,\mu)  &
  P_{vv}(k,\mu) + \frac{\sigma^2}{n}
   \end{bmatrix},
\label{cov:eq}
\end{equation}
and $\sigma \approx (\frac{5}{\ln{10}} \frac{1+z}{z})^{-1} \sigma_M$.
The dependences of $\gamma$ and other parameters enter through $fD$ in the relations $P_{vv}\propto (fD\mu)^2$, the SN~Ia host-galaxy count overdensity
power spectrum $P_{\delta \delta }\propto (bD + fD\mu^2)^2$, and the galaxy-velocity cross-correlation $P_{vg}
\propto  (bD + fD\mu^2)fD$.  Details
on the assumptions made for the calculation can be found in \cite{2019BAAS...51c.140K}.

ZTF2 and TAIPAN are near-term surveys that will measure peculiar velocities.  Both have roughly
 $z_{\text{max}}=0.09$ and $\Omega = 2\pi$.
Uncertainties in $\gamma$ for surveys with this depth and solid-angle coverage 
are shown as a function of number of sources $N$ and $\sigma_M$ in Figure~\ref{surface:fig}. The positions of ZTF2 and TAIPAN are marked
in the Figure, with the latter showing both  a conservative $\sigma_M=0.12$~mag and the other for an aggressive $\sigma_M=0.08$~mag.
ZTF2 and TAIPAN lie in opposite ends of the diagram, there are small numbers of ZTF2 SNe with precise distances
whereas there are many TAIPAN Fundamental Plane galaxies with imprecise distances. 
Conservative ZTF2 and TAIPAN
are projected to give similar precisions $\sigma_ \gamma = 0.054$, whereas  the lower intrinsic magnitude dispersion  gives
$\sigma_ \gamma = 0.042$.  
Recall that the difference in $\gamma$  between GR and the $f(R)$ and DGP gravity is 0.13, meaning that these surveys
can already distinguish between these models to $2-3 \sigma$.

\begin{figure}
\centering
\includegraphics[width=0.7\textwidth]{src/surface1.pdf}
\caption{Uncertainties in $\gamma$ for surveys  with $z_{\text{max}}=0.09$ and $\Omega = 2\pi$.
are shown as a function of number of sources $N$ and $\sigma_M$.  The positions of ZTF2 and TAIPAN are marked,
with the latter showing both  a conservative $\sigma_M=0.12$~mag and the other for an aggressive $\sigma_M=0.08$~mag.
\label{surface:fig}}
\end{figure}

Being in different hemispheres, the two surveys
complementarily survey different parts of sky, meaning that their two independent results can be
combined  quadratically to produce a reduced joint uncertainty.  An important distinction between the surveys is
that TAIPAN will observe almost all available Fundamental Plane galaxies in the volume, meaning that no additional observing can
increase the number density $n$ to
move the TAIPAN point further rightwards in the diagram.  On the other hand, the number density of supernova increases
linearly with time, meaning that since ZTF2 is not yet sample variance limited there is continued room for improvement with longer surveys.


The conservative and aggressive values for intrinsic magnitude dispersion are related to the quality and variety of available supernova
data
obtained.  ZTF2 light-curves alone are projected to yield $0.12$~mag distance modulus uncertainties per supernova.
Supplemental supernova follow-up data, say in the near-infrared and/or
spectroscopy, from other telescopes can reduce the uncertainty to at least $\sigma_M=0.08$~mag.  The requirements
and benefits of data beyond the imaging survey are discussed further in \S.

A long-term supernova peculiar-velocity survey can be done with SNe~Ia discovered by LSST.
As a 10-year survey, LSST generates higher supernova number densities to fainter limiting magnitude than the ZTF2 survey,
making possible significantly improved constraints on the growth index.
All the proposed LSST surveys have complete SN~Ia discovery out to $z=0.3$.
The expected distance uncertainties derived from LSST light curves vary greatly depending on observing strategy, but at best
is expected to be $\sigma_M=0.12$~mag.    As with the ZTF2 survey, even lower magnitude uncertainties
can be achieved with supplemental data.

Uncertainties in $\gamma$ for surveys with a 10-year duration and  and $\Omega=2\pi$ sky coverage 
are shown as a function of number of limiting  redshift $z_{max}$ and $\sigma_M$ in Figure~\ref{lsst:fig}.
This is applicable to LSST.
The redshift depth afforded by LSST provides real improvement relative to the shallower ZTF2 survey.
After 10 years, the region with $z_{max}<0.1$ have sensitivity to  $\sigma_\gamma$ that is weakly sensitive to $\sigma_M$, 
which reflects the relatively strong sample variance in these small survey volume.  Better growth index constraints
are achieved by going to $z_{max}>0.1$.  However, the differential benefit decreases quickly with increasing redshift;
the increase in volume does not make up for the larger velocity uncertainties.

\begin{figure}
\centering
\includegraphics[width=0.7\textwidth]{src/surface2.pdf}
\caption{Uncertainties in $\gamma$ for surveys with a 10-year duration and  and $\Omega=2\pi$ sky coverage 
are shown as a function of number of limiting  redshift $z_{max}$ and $\sigma_M$.
\label{lsst:fig}}
\end{figure}

%Given these  advances, supernovae discovered by wide-field searches in the next decade will 
% be able to tightly constrain the growth of structure in the low-redshift Universe.
%For example, 
%over the course of a decade a SN survey relying on LSST discoveries 
%plus spectroscopic redshifts can produce  4--14\% uncertainties in $fD$ in $0.05$ redshift bins from $z=0$ to 0.3, cumulatively giving
%2.2\% uncertainty on $fD$  within this interval, where at
%$0<z<0.2$ most of the probative power comes from peculiar velocities and at higher redshifts from RSD  \cite{2017ApJ...847..128H}.
%
%\section{Testing Gravity with Peculiar Velocity Surveys}
%While the growth rate $fD$ can be used to test several aspects of physics beyond the standard cosmological model (e.g.\ dark matter clustering, dark energy evolution), our scientific interest is in probing gravity,
%so here we focus  on the growth index $\gamma$.
%To illustrate the distinction,
% $\frac{d(\ln{fD})}{d\gamma} = \ln\Omega_M + \int \Omega_M^{\gamma} \ln{\Omega_M}\, d\ln{a} \approx -1.68, -0.75,-0.37$ at $z=0,0.5,1.0$ respectively
% in $\Lambda$CDM;  two surveys with the same fractional precision  in $fD$ will have different precision in $\gamma$,
% with the one at lower redshift providing the tighter constraint. 
% In this section, we demonstrate that peculiar velocity surveys in the upcoming decade can measure $\gamma$ precisely
%for a range of survey-parameter choices.
% 
%
%We project uncertainties  on the growth index, $\sigma_\gamma$ for a suite of idealized surveys
%using a Fisher matrix analysis similar to that of \cite{2017ApJ...847..128H, 2017MNRAS.464.2517H}
%(there is an alternative approach using an estimator for the mean pairwise velocity
%\cite{2011PhRvD..83d3004B}).
%The ``cross-correlation'' analysis incorporates both galaxy overdensities and peculiar velocities.

%The parameter dependence enters through $fD$ in the relations $P_{vv}\propto (fD\mu)^2$, the SN~Ia host-galaxy count overdensity
%power spectrum $P_{\delta \delta }\propto (bD + fD\mu^2)^2$, and the galaxy-velocity cross-correlation $P_{vg}
%\propto  (bD + fD\mu^2)fD$, where $b$ is the galaxy bias and $\mu\equiv \cos{(\hat{k} \cdot \hat{r})}$ where $\hat{r}$ is the direction of
%the line of sight.  While the $bD$ term does contain information on $\gamma$, its constraining power is not used here.
%Both $f$ and $D$ depend on $\Omega_M=\frac{\Omega_{M_0}}{\Omega_{M_0} + (1-\Omega_{M_0})a^3}$.
%The uncertainty in $\gamma$ is $\sigma_\gamma = \sqrt{\left(F^{-1}\right)_{\gamma\gamma}}$.
%Non-GR models may  also predict a change in the scale-dependence of the growth or non-constant $\gamma$, such observations provide additional leverage in probing gravity
%but are not considered here.

%The uncertainty  $\sigma_\gamma$ of a survey depends on its solid angle $\Omega$, depth
%given by the comoving distance out to the maximum redshift $r_{\rm max}=r(z_{\rm max})$, duration
%$t$ through $n=\epsilon \phi t$ where $\phi$ is the observer-frame SN~Ia rate
%and $\epsilon$ is the sample-selection efficiency, and
%the intrinsic SN~Ia magnitude dispersion 
%through the resulting peculiar velocity intrinsic dispersion $\sigma \approx (\frac{5}{\ln{10}} \frac{1+z}{z})^{-1} \sigma_M$.
%
%
%We consider SN peculiar velocity surveys for a range of redshift depths $z_{\rm max}$ for durations of $t=2$ and 10~years.
%The other survey parameters
%$\Omega=3\pi$, $\epsilon=0.65$, $\sigma_M=0.08$~mag are fixed.
%The $k$-limits are  taken to be $k_{\rm min} = \pi/r_{\rm max}$  and 
%$k_{\rm max} = 0.1$~$h$\,Mpc$^{-1}$.
%A minimum distance $r_{\rm min}=r(z=0.01)$ is imposed as our analysis assumes that peculiar velocities are significantly  smaller than
%the cosmological redshift.
%The sample-selection efficiency $\epsilon$ is redshift-independent, i.e.\ the native redshift distribution is not sculpted.
%The input bias of SN~Ia host galaxies is set as $b=1.2$.  An independent measurement of $\Omega_{M_0}=0.3\pm0.005$
%is included and is a non-trivial contributor to the $\gamma$ constraint.  Number densities are taken to be
%direction-independent, neglecting the slight declination-dependence of SN-survey time windows
%
%All the surveys considered  provide meaningful tests of gravity.
%The projected uncertainty in $\gamma$ achieved by the suite of surveys are shown in Figure~\ref{var:fig}.
%The primary result is for the cross-correlation analysis that uses overdensities (RSD), peculiar velocities, and their cross-correlations.
%The short and shallow,  2-year, $z_{\rm max}=0.11$ survey has $\sigma_\gamma \sim 0.038$,
%which can distinguish  between  General Relativity, $f(R)$,  and DGP gravities at the $>3\sigma$ level.
%The 10-year survey performance asymptotes at $z_{\rm max} \sim0.2$ at a precision of  $\sigma_\gamma \sim 0.01$.
%Figure~\ref{var:fig} also shows uncertainties based on two other analyses, one that only uses peculiar velocities, and
%one that combines independent RSD and peculiar velocity results.
%Peculiar velocities alone
%account for much of the probative power of the surveys. RSD alone do not provide significant constraints.
%However, considering RSD and velocity cross-correlations decreases $\sigma_\gamma$ by  $\sim 20$\%.
%The implication is that there are important
%$k$-modes that are  sample variance limited either in  overdensity and/or peculiar velocity who benefit from the
%sample-noise suppression engendered by cross-correlations.

%\begin{figure}
%\centering
%\includegraphics[width=0.49\textwidth]{src/var.png}
%\caption{The projected uncertainty in $\gamma$, $\sigma_\gamma$, achieved by two-year (red) and ten-year (black) SN~Ia surveys of varying depth $z_{\rm max}$.
%For each survey uncertainties are based on three types of analyses:  using only peculiar velocities (dashed); using
%both RSD and peculiar velocities independently (dotted);  using
%both RSD, peculiar velocities, and their cross-correlation (solid).
%\label{var:fig}}
%\end{figure}

\section{LBL Plan for a Peculiar Velocity Program}
The projections for measuring $\gamma$ with ZTF2 and LSST SN~Ia discoveries
show the power of peculiar velocities surveys at low ($z<0.2$) redshift.
This unmatched new tracer opens a new window with which to address
the cosmological puzzles we are faced with today.

LBL proposes to pursue the following course of research for the upcoming decade.

\subsection{Near Term: ZTF2}

Program has requisite data of which imaging survey is required but insufficient.

For the LBL peculiar-velocity program, we propose a near-term ZTF2.  Long-term LSST.

Why ZTF2?


It is ready to go now.

Relatively cheap, infrastructure in place.

Head start on developing skills for LSST.

Northern Hemisphere.

Already a scientifically interesting constraint.

\subsection{Long Term: LSST}

\subsection{Complementarity with High-$z$ redshift surveys}


\section{Buy In}



Joining ZTFII requires buy-in.  Here is a list of things that LBL could bring to the table.
Cash
DOE.  Thought it will be hard to convince Kathy
Foundation money
NERSC Computing
Offer ZTFII Data management at NERSC
Couple this with SkyPortal and UCB
UH-88 + SNIFS
LBL negotiates with UH for UH-88 time to follow-up ZTFII discoveries
LBL responsible for SNIFS data management
Foundation funding for facility support, SNIFS refurbishment
Domain expertise being developed now
Intercollaboration DESI-ZTFII density plus peculiar-velocity analysis?
Create a DESI2 program beneficial to ZTFII.   In the last two years of DESI there will be opportunity for pilot studies.
LSST+
A new Project outside the confines of DESC for which LBL could have management responsibilities
Provide ZTFII observations on instruments part of the new Project
UH-88 + SNIFS
New French spectrograph mounted at ESO
Network of ?identical? spectrographs distributed around the world
LSST-DESC
LBL could support ZTF2 international partners in exchanging ZTF catalogs for LSST buy-in
DESC process has started
DESC pipeline scientists contributing to Peculiar Velocity pipeline
More naturally this would be a DESC-ZTF2 arrangement without LBL playing a special role

A selfish objective is to have money flow through LBL
A new Peculiar Velocity experiment outside of DESC
Already requested for a ramp up of DOE-supported SN postdocs at LBL

\subsection{Through DESI}
DESI

We have already built a collaboration between DESI and the ZTF SN Ia Team
DESI provides early access to spectra of transient host-galaxies taken as part of its surveys, particularly relevant is the BGS.
DESI prioritizes observations of survey objects of interest to ZTF


Galactic time.

\section{Larger Community}

\section{Conclusions}

%SNe~Ia are already powerful probes of the homogeneous cosmological expansion of the Universe.  
In the next decade,
the high number of SN discoveries together with improved precision in their distance precisions will make $z<0.3$ SNe~Ia, more so than
galaxies,  powerful probes of gravity through their effect  on the growth of structure.  Different survey strategies can be adopted to take advantage of these
supernovae, and in this White Paper we present a formalism and code (available
at \url{http:tiny.cc/PVScience})
by which their scientific merits can be assessed and present results for a range of options.
%A peculiar velocity survey must discover, classify, and get distances of SNe~Ia, as well as obtain redshifts of their host galaxies.
%Choices in survey design impact the precision obtained on the growth index $\gamma$.  

No other probe of growth of structure or tracer of peculiar velocity can alone provide comparable precision on  $\gamma$ in the next decade.
At low redshift, the RSD measurement is quickly sample variance limited (as are the planned DESI BGS and 4MOST surveys) making peculiar velocities the only 
precision probe of $fD$.
TAIPAN and a TAIPAN-like DESI BGS will be able to measure FP distances for nearly all usable nearby galaxies, so at low-$z$ the Fundamental Plane peculiar-velocity
technique will  saturate at a level that is not competitive with a  2-year SN survey.

Combined low-redshift peculiar velocity and high-redshift RSD $fD$ measurements are highly complementary as together they probe the
$\gamma$-dependent shape of $fD(z)$ (not just its normalization) and potential scale-dependent influence of gravitational models, since low-
and high-redshift surveys are weighted by lower and higher $k$-modes respectively.
SN~Ia peculiar velocity surveys are of the highest scientific
interest and we encourage
the community to develop aggressive surveys in the pursuit of testing  General Relativity and probing gravity. 



\bibliographystyle{plain}
\bibliography{/Users/akim/Documents/alex}
\end{document}  